% author:   sam tenka
% change:   2022-05-27
% create:   2022-05-11

%==============================================================================
%====  0.  DOCUMENT SETTINGS  ================================================
%==============================================================================

%~~~~~~~~~~~~~~~~~~~~~~~~~~~~~~~~~~~~~~~~~~~~~~~~~~~~~~~~~~~~~~~~~~~~~~~~~~~~~~
%~~~~~~~~~~~~~  0.0. About this Exposition  ~~~~~~~~~~~~~~~~~~~~~~~~~~~~~~~~~~~

%---------------------  0.0.0. page geometry  ---------------------------------
\documentclass[11pt, justified]{tufte-book}
\geometry{
  left           = 0.90in, % left margin
  textwidth      = 4.95in, % main text block
  marginparsep   = 0.15in, % gutter between main text block and margin notes
  marginparwidth = 2.30in, % width of margin notes
                 % 0.20in  % width from margin to edge
}

%---------------------  0.0.1. math packages  ---------------------------------
\newcommand\hmmax{0} % to allow for more fonts 
\newcommand\bmmax{0} % to allow for more fonts
\usepackage{amsmath, amssymb, amsthm, mathtools}
\usepackage{bm}
\usepackage{euler}

\usepackage{array}   % for \newcolumntype macro
\newcolumntype{L}{>{$}l<{$}} % math-mode version of "l" column type
\newcolumntype{C}{>{$}c<{$}} % math-mode version of "c" column type
\newcolumntype{R}{>{$}r<{$}} % math-mode version of "r" column type

%---------------------  0.0.2. graphics packages  -----------------------------
\usepackage{graphicx, xcolor}
\usepackage{float, capt-of}

%---------------------  0.0.3. packages for fancy text  -----------------------
\usepackage{enumitem}\setlist{nosep}
\usepackage{listings}
\usepackage{xstring}
\usepackage{fontawesome5}

%---------------------  0.043. colors  ----------------------------------------
\definecolor{mblu}{rgb}{0.05, 0.35, 0.70} \newcommand{\blu}{\color{mblu}}
\definecolor{mbre}{rgb}{0.30, 0.45, 0.60} \newcommand{\bre}{\color{mbre}}
\definecolor{mbro}{rgb}{0.60, 0.05, 0.05} \newcommand{\bro}{\color{mbro}}
\definecolor{mcya}{rgb}{0.10, 0.45, 0.45} \newcommand{\cya}{\color{mcya}}
\definecolor{mgre}{rgb}{0.55, 0.55, 0.50} \newcommand{\gre}{\color{mgre}}
\definecolor{mgrn}{rgb}{0.15, 0.65, 0.05} \newcommand{\grn}{\color{mgrn}}
\definecolor{mred}{rgb}{0.90, 0.05, 0.05} \newcommand{\red}{\color{mred}}

%~~~~~~~~~~~~~~~~~~~~~~~~~~~~~~~~~~~~~~~~~~~~~~~~~~~~~~~~~~~~~~~~~~~~~~~~~~~~~~
%~~~~~~~~~~~~~  0.1. Headers and References  ~~~~~~~~~~~~~~~~~~~~~~~~~~~~~~~~~~

%---------------------  0.1.0. intra-document references  ---------------------
\newcommand{\offour}[1]{
    {\tiny \raisebox{0.04cm}{\scalebox{0.9}{$\substack{
        \IfSubStr{#1}{0}{{\blacksquare}}{\square}   
        \IfSubStr{#1}{1}{{\blacksquare}}{\square} \\ 
        \IfSubStr{#1}{2}{{\blacksquare}}{\square}   
        \IfSubStr{#1}{3}{{\blacksquare}}{\square}   
    }$}}}%
}

\newcommand{\offourline}[1]{
    {\tiny \raisebox{0.04cm}{\scalebox{0.9}{$\substack{
        \IfSubStr{#1}{0}{{\blacksquare}}{\square}   
        \IfSubStr{#1}{1}{{\blacksquare}}{\square}
        \IfSubStr{#1}{2}{{\blacksquare}}{\square}   
        \IfSubStr{#1}{3}{{\blacksquare}}{\square}   
    }$}}}%
}
\newcommand{\notesam}[1]{{\blu \textsf{#1}}}
\newcommand{\attn}[1]{{\bro \textsf{#1}}}
\newcommand{\attnsam}[1]{{\red \textsf{#1}}}

\newcommand{\blarr}{\hspace{-0.15cm}${\bro \leftarrow}\,$}
\newcommand{\bcirc}{${\bro ^\circ}$}

\newcounter{footprintssofar}
\setcounter{footprintssofar}{90}
\newcommand{\plainfootprint}{{\bro \rotatebox{\value{footprintssofar}}{\faIcon{shoe-prints}}}\setcounter{footprintssofar}{\value{footprintssofar}+30} }
\newcommand{\footprint}{\marginnote{\plainfootprint} }

%---------------------  0.1.1. table of contents helpers  ---------------------
\newcommand{\phdot}{\phantom{.}}

%---------------------  0.1.2. section headers  -------------------------------
\newcommand{\samtitle} [1]{
  \par\noindent{\Huge \sf \blu #1}
  \vspace{0.4cm}
}

\newcommand{\samquote} [2]{
    \marginnote[-0.4cm]{\begin{flushright}
    \scriptsize
        \gre {\it #1} \\ --- #2
    \end{flushright}}
}

\newcommand{\samsection} [1]{
  \vspace{0.5cm}
  \par\noindent{\LARGE \sf \blu #1}
  \vspace{0.1cm}\par
}

\newcommand{\samsubsection}[1]{
  \vspace{0.3cm}
  \par\noindent{\Large \sf \bre #1}
  \vspace{0.1cm}\par
}

\newcommand{\samsubsubsection}[1]{
   \vspace{0.1cm}
   \par\noindent{\hspace{-2cm}\normalsize \sc \gre #1} ---
}

%---------------------  0.1.3. clear the bibliography's header  ---------------
\usepackage{etoolbox}
\patchcmd{\thebibliography}{\section*{\refname}}{}{}{}

%~~~~~~~~~~~~~~~~~~~~~~~~~~~~~~~~~~~~~~~~~~~~~~~~~~~~~~~~~~~~~~~~~~~~~~~~~~~~~~
%~~~~~~~~~~~~~  0.2. Math Symbols and Blocks  ~~~~~~~~~~~~~~~~~~~~~~~~~~~~~~~~~

%---------------------  0.2.0. general math operators  ------------------------
\newcommand{\scirc}{\mathrel{\mathsmaller{\mathsmaller{\mathsmaller{\circ}}}}}
\newcommand{\cmop}[2]{{(#1\!\to\!#2)}}

%---------------------  0.2.1. probability symbols  ---------------------------
\newcommand{\KL}{\text{KL}}
\newcommand{\EN}{\text{H}}
\newcommand{\note}[1]{{\blu \textsf{#1}}}

%---------------------  0.2.2. losses averaged in various ways  ---------------
\newcommand{\Ein}  {\text{trn}_{\sS}}
\newcommand{\Einb} {\text{trn}_{\check\sS}}
\newcommand{\Einc} {\text{trn}_{\sS\sqcup \check\sS}}
\newcommand{\Egap} {\text{gap}_{\sS}}
\newcommand{\Eout} {\text{tst}}

%---------------------  0.2.3. double-struck and caligraphic upper letters  ---
\newcommand{\Aa}{\mathbb{A}}\newcommand{\aA}{\mathcal{A}}
\newcommand{\Bb}{\mathbb{B}}\newcommand{\bB}{\mathcal{B}}
\newcommand{\Cc}{\mathbb{C}}\newcommand{\cC}{\mathcal{C}}
\newcommand{\Dd}{\mathbb{D}}\newcommand{\dD}{\mathcal{D}}
\newcommand{\Ee}{\mathbb{E}}\newcommand{\eE}{\mathcal{E}}
\newcommand{\Ff}{\mathbb{F}}\newcommand{\fF}{\mathcal{F}}
\newcommand{\Gg}{\mathbb{G}}\newcommand{\gG}{\mathcal{G}}
\newcommand{\Hh}{\mathbb{H}}\newcommand{\hH}{\mathcal{H}}
\newcommand{\Ii}{\mathbb{I}}\newcommand{\iI}{\mathcal{I}}
\newcommand{\Jj}{\mathbb{J}}\newcommand{\jJ}{\mathcal{J}}
\newcommand{\Kk}{\mathbb{K}}\newcommand{\kK}{\mathcal{K}}
\newcommand{\Ll}{\mathbb{L}}\newcommand{\lL}{\mathcal{L}}
\newcommand{\Mm}{\mathbb{M}}\newcommand{\mM}{\mathcal{M}}
\newcommand{\Nn}{\mathbb{N}}\newcommand{\nN}{\mathcal{N}}
\newcommand{\Oo}{\mathbb{O}}\newcommand{\oO}{\mathcal{O}}
\newcommand{\Pp}{\mathbb{P}}\newcommand{\pP}{\mathcal{P}}
\newcommand{\Qq}{\mathbb{Q}}\newcommand{\qQ}{\mathcal{Q}}
\newcommand{\Rr}{\mathbb{R}}\newcommand{\rR}{\mathcal{R}}
\newcommand{\Ss}{\mathbb{S}}\newcommand{\sS}{\mathcal{S}}
\newcommand{\Tt}{\mathbb{T}}\newcommand{\tT}{\mathcal{T}}
\newcommand{\Uu}{\mathbb{U}}\newcommand{\uU}{\mathcal{U}}
\newcommand{\Vv}{\mathbb{V}}\newcommand{\vV}{\mathcal{V}}
\newcommand{\Ww}{\mathbb{W}}\newcommand{\wW}{\mathcal{W}}
\newcommand{\Xx}{\mathbb{X}}\newcommand{\xX}{\mathcal{X}}
\newcommand{\Yy}{\mathbb{Y}}\newcommand{\yY}{\mathcal{Y}}
\newcommand{\Zz}{\mathbb{Z}}\newcommand{\zZ}{\mathcal{Z}}

%---------------------  0.2.4. sans serif and frak lower letters  -------------
\newcommand{\sfa}{\mathsf{a}}\newcommand{\fra}{\mathcal{a}}
\newcommand{\sfb}{\mathsf{b}}\newcommand{\frb}{\mathcal{b}}
\newcommand{\sfc}{\mathsf{c}}\newcommand{\frc}{\mathcal{c}}
\newcommand{\sfd}{\mathsf{d}}\newcommand{\frd}{\mathcal{d}}
\newcommand{\sfe}{\mathsf{e}}\newcommand{\fre}{\mathcal{e}}
\newcommand{\sff}{\mathsf{f}}\newcommand{\frf}{\mathcal{f}}
\newcommand{\sfg}{\mathsf{g}}\newcommand{\frg}{\mathcal{g}}
\newcommand{\sfh}{\mathsf{h}}\newcommand{\frh}{\mathcal{h}}
\newcommand{\sfi}{\mathsf{i}}\newcommand{\fri}{\mathcal{i}}
\newcommand{\sfj}{\mathsf{j}}\newcommand{\frj}{\mathcal{j}}
\newcommand{\sfk}{\mathsf{k}}\newcommand{\frk}{\mathcal{k}}
\newcommand{\sfl}{\mathsf{l}}\newcommand{\frl}{\mathcal{l}}
\newcommand{\sfm}{\mathsf{m}}\newcommand{\frm}{\mathcal{m}}
\newcommand{\sfn}{\mathsf{n}}\newcommand{\frn}{\mathcal{n}}
\newcommand{\sfo}{\mathsf{o}}\newcommand{\fro}{\mathcal{o}}
\newcommand{\sfp}{\mathsf{p}}\newcommand{\frp}{\mathcal{p}}
\newcommand{\sfq}{\mathsf{q}}\newcommand{\frq}{\mathcal{q}}
\newcommand{\sfr}{\mathsf{r}}\newcommand{\frr}{\mathcal{r}}
\newcommand{\sfs}{\mathsf{s}}\newcommand{\frs}{\mathcal{s}}
\newcommand{\sft}{\mathsf{t}}\newcommand{\frt}{\mathcal{t}}
\newcommand{\sfu}{\mathsf{u}}\newcommand{\fru}{\mathcal{u}}
\newcommand{\sfv}{\mathsf{v}}\newcommand{\frv}{\mathcal{v}}
\newcommand{\sfw}{\mathsf{w}}\newcommand{\frw}{\mathcal{w}}
\newcommand{\sfx}{\mathsf{x}}\newcommand{\frx}{\mathcal{x}}
\newcommand{\sfy}{\mathsf{y}}\newcommand{\fry}{\mathcal{y}}
\newcommand{\sfz}{\mathsf{z}}\newcommand{\frz}{\mathcal{z}}

%---------------------  0.2.5. math environments  -----------------------------
\newtheorem*{qst}{Question}
\newtheorem*{thm}{Theorem}
\newtheorem*{lem}{Lemma}
% ...
\theoremstyle{definition}
\newtheorem*{dfn}{Definition}

%==============================================================================
%=====  1.  PROLOGUE  =========================================================
%==============================================================================

\begin{document}
\samtitle{python recitation (optional 6.86x notes)}

      \marginnote{%
        %These notes overlap with what we'll cover in recitation.  But
        Recitation will have more coding practice and less detail than
        these notes.  We'll skip whole passages of these notes during
        recitation.
        %
        Footprints --- \plainfootprint --- roughly indicate our pacing: we will
        linger on the points near each footprint for three to seven
        minutes of recitation time.  
        %
        %A recitation might have eighteen footprints.
      }

      \attn{You do not need to read these notes at all} to get an A
      in this course; conversely, \attn{you may not cite these notes} when
      solving homework or exams.
       
  \samsection{appendix: python programming}

    \samsubsection{python setup}
      \samquote{
        Displace one note and there would be diminishment, displace one phrase
        and the structure would fall.
      }{antonio salieri, on wolfgang mozart's music, as untruthfully portrayed in \emph{Amadeus}}

      \samsubsubsection{what's python?}
        Python is a popular programming language.  Its heart is the
        \textbf{Python interpreter}, a computer program that takes a plain old
        text file such as this two-liner\bcirc\marginnote{%
          \blarr The instruction \texttt{print} just displays some text
          onto our screen.  For example, the first line of this program 
          displays \texttt{hello, world!} onto our screen.  This instruction
          doesn't rely on or activate any ink-on-paper printing machines.
        } ---
        \begin{lstlisting}[language=Python, basicstyle=\footnotesize\ttfamily]
          print('hello, world!')
          print('cows and dogs are fluffy')
        \end{lstlisting}
        --- and executes the instructions contained in that text file.  The
        jargon is that these textual instructions are \textbf{source code}.

        The instructions have to be in a certain, extremely rigid format in
        order for the interpreter to understand and execute them.  That's why
        they call Python a \emph{language}: it has its own rigid grammar
        and its own limited vocabulary.  If the
        interpreter encounters incorrectly formatted instructions --- if even a
        single punctuation mark is missing or a --- the interpreter will display a
        bit of text in addition to the word \texttt{Error} and immediately
        after abort its mission.\bcirc\marginnote{%
          \blarr Adventure boldly when learning Python!  It might feel
          catastrophic when you encounter an error and the interpreter
          `dies'.  But (unless you go out of your way to use special
          instructions we won't teach in class) these errors won't hurt your
          computer.  There aren't any lasting effects. 
          %
          So \textbf{errors are hints, not penalties}. 
          If you encounter an
          error, just modify your instructions to address that error, then run
          the interpreter again.
          %
          Engage in a fast feedback cycle (\emph{I'll try this... error...
          okay how about this...  different error... hmm let's think...}) 
          to learn to program well.
        }

        We'll use Python in 6.86x to instruct our computer to analyze and learn
        from data.  The gigantic project of instructing a computer to learn is
        a bit like teaching a person by mail.  We never see them: we only
        exchange words back and forth.  They have never seen a horse, and yet
        we want to teach them to distinguish horses from zebras from donkeys
        from tapirs from rhinos.  Though severely limited in their vocabulary
        and their ability to appreciate analogies and similarities, they are
        extraordinarily meticulous, patient, and efficient. 
        %
        That's what programming will be like.

        At this point, you might have several questions:
        \begin{description}
          \item[\textbf{Picking up the pen:}] How do I install and use the Python
               interpreter? 
          \item[\textbf{Writing sentences:}] What syntax rules must my
               instructions obey in order for the interpreter to understand
                that I want it do such-and-such task? 
          \item[\textbf{Composing essays:}] How do I organize large programs?
          \item[\textbf{Teaching via mail:}] What instructions make
               make the computer learn from data?
        \end{description}
        This and the next three subsections address these four
        questions in turn.

      \samsubsubsection{which things we'll set up}
        Let's set up Python by installing the Python interpreter.
        %
        Actually, I should say \emph{a} Python interpreter: each of the many
        software tools we'll use in 6.86x has a zillion versions; it can get
        confusing tracking which versions coexist and which clash.  We will use
        \textbf{Python version 3.8} throughout 6.86x.

        Beyond the Python interpreter, there is an ecosystem of useful tools:
        machine learning modules that help us avoid reinventing the wheel; a
        rainbow of feature-rich text editors specialized for writing Python code;
        and a package manager called \texttt{conda} that eases the logistics
        nightmare of coordinating the versions we have of each tool.

      \samsubsubsection{setup on windows}
        \attnsam{Mohamed, please fill this out}
        \attnsam{(mention windows 10 and higher's `linux subsystem?')}

      \samsubsubsection{setup on macOS}
        \attnsam{Karene, plese fill this out}

      \samsubsubsection{setup on linux}
        I'll assume we're working with Ubuntu 20.04.  If you're on a
        different Linux, then similar steps should work --- google search to
        figure out how.  

      \samsubsubsection{checking the setup}
        Let's create a new plain text file containing this single line:\bcirc\marginnote{%
          \blarr This line contains \textbf{Python source code}.  When we include
          Python source code in these notes, we will format it for readability,
          e.g.\ by making key parts of it bold.  However, this depiction of
          source code is supposed to represent \emph{plain text}.
        }
        \begin{lstlisting}[language=Python, basicstyle=\footnotesize\ttfamily]
          print('hello, world!')
        \end{lstlisting}
        We can name the file whatever we want --- say, \texttt{greetings.py}.
        Then in your terminal, enter this command:
        \begin{lstlisting}[basicstyle=\footnotesize\ttfamily]
          python3 greetings.py
        \end{lstlisting}
        A new line should appear in your terminal:
        \begin{lstlisting}[basicstyle=\footnotesize\ttfamily]
          hello, world!
        \end{lstlisting}
        After that line should be another shell prompt.

        Now append three lines to the file \texttt{greetings.py} so that its
        contents look like:
        \begin{lstlisting}[language=Python, basicstyle=\footnotesize\ttfamily]
          print('hello, world!')
          fahr = int(input('please enter a number... '))
          celc = int((fahr - 32.0) * 5.0/9.0)
          print('{} fahrenheit is roughly {} celcius'.format(fahr, celc))
        \end{lstlisting}
        Again, enter the following command in your terminal:
        \begin{lstlisting}[basicstyle=\footnotesize\ttfamily]
          python3 greetings.py
        \end{lstlisting}
        Two new lines should appear in your terminal:
        \begin{lstlisting}[basicstyle=\footnotesize\ttfamily]
          hello, world!
          please enter a number...
        \end{lstlisting}
        Enter some number --- say, $72$; a new line should appear in your
        terminal so that the overall session looks something like:
        \begin{lstlisting}[basicstyle=\footnotesize\ttfamily]
          hello, world!
          please enter a number... 72
          72 fahrenheit is roughly 22 celcius 
        \end{lstlisting}
        After that line should be another shell prompt.

        If you as did the above something different from the predicted lines
        appeared, that means something is amiss with Python setup. 
        Please let us know in this case!  We can try to help fix. 

    \newpage
    \samsubsection{python basics}
      % use-mention distinction; syntax-semantic.s  Asimov ROBOT????
      % TODO: figure out what I meant by the comment line directly above
      \marginnote{%
        Once you learn how to program, I highly recommend reading Edsger
        Dijkstra's \emph{A Discipline of Programming}.  That book contains many
        beautiful viewpoints that made me a better programmer.
      }

      \samsubsubsection{output and sequencing}
        Try this one-liner:
        \begin{lstlisting}[language=Python, basicstyle=\footnotesize\ttfamily]
          print('hello!')
        \end{lstlisting}
        Now try this:
        \begin{lstlisting}[language=Python, basicstyle=\footnotesize\ttfamily]
          print(686)
        \end{lstlisting}
        We execute multiple instructions in sequence by placing them on
        successive lines:
        \begin{lstlisting}[language=Python, basicstyle=\footnotesize\ttfamily]
          print('hello!')
          print(6)
          print(86)
          print('goodbye!')
        \end{lstlisting}

      \samsubsubsection{state and assignment}
        Try this two-liner:
        \begin{lstlisting}[language=Python, basicstyle=\footnotesize\ttfamily]
          x = 686
          print(x)
        \end{lstlisting}
        And this three-liner:
        \begin{lstlisting}[language=Python, basicstyle=\footnotesize\ttfamily]
          x = 686
          x = 123
          print(x)
        \end{lstlisting}
        Does the following print the same thing?
        \begin{lstlisting}[language=Python, basicstyle=\footnotesize\ttfamily]
          x = 123
          x = 686
          print(x)
        \end{lstlisting}
        What do you predict this does:
        \begin{lstlisting}[language=Python, basicstyle=\footnotesize\ttfamily]
          x = 123
          print(x)
          x = 686
          print(x)
        \end{lstlisting}
        %
        Above, \texttt{x} names a \textbf{variable}.  A variable stores some
        value at each point in time; it might store different values at
        different times.  We make a variable just by mentioning it in an
        \textbf{assignment} instruction such as \texttt{x=123}.   We get to
        choose each variable's name however we'd like; a variable may have a
        one-letter name or a longer name:\bcirc\marginnote{%
          \blarr Naming rules: the first character must belong to the English
          alphabet (so there are $26$ lower case choices plus $26$ upper case
          choices).  Each subsequent character is either an underscore (that's
          the symbol that looks like \texttt{\_}), a digit (\texttt{0} through \texttt{9}),
          or a member of the English alphabet.
          %
          The following names are forbidden, since they already mean
          something else in Python:
            \texttt{assert},
            \texttt{bool},
            \texttt{break},
            \texttt{chr},
            \texttt{class},
            \texttt{continue},
            \texttt{def},
            \texttt{dict},
            \texttt{elif},
            \texttt{else},
            \texttt{enumerate},
            \texttt{except},
            \texttt{float},
            \texttt{for},
            \texttt{if},
            \texttt{import},
            \texttt{input},
            \texttt{int},
            \texttt{lambda},
            \texttt{len} 
            \texttt{list},
            \texttt{max},
            \texttt{min},
            \texttt{ord},
            \texttt{print},
            \texttt{range},
            \texttt{round},
            \texttt{set},
            \texttt{str},
            \texttt{sum},
            \texttt{try},
            \texttt{type},
            \texttt{while},
            \texttt{with},
            \texttt{zip}.
          %
          For a full list of forbidden names, google search \emph{list of
          Python keywords and built-in functions}.
        }
        \begin{lstlisting}[language=Python, basicstyle=\footnotesize\ttfamily]
          my_favorite_number = 123
          print(my_favorite_number)
          my_favorite_number = 686
          print(my_favorite_number)
        \end{lstlisting}
        %
        We can use multiple \textbf{variables}:
        \begin{lstlisting}[language=Python, basicstyle=\footnotesize\ttfamily]
          x = 123
          y = 686
          print(x)
        \end{lstlisting}
        What do you think the following prints?
        \begin{lstlisting}[language=Python, basicstyle=\footnotesize\ttfamily]
          x = 123
          y = 686
          x = y
          print(x)
          print(y)
        \end{lstlisting}

      \samsubsubsection{arithmetic}
        We can put math on the right hand side of assignments:\marginnote{%
          \attnsam{variable cube; assignment as shearing}
        }
        \begin{lstlisting}[language=Python, basicstyle=\footnotesize\ttfamily]
          x = 123
          y = x*2 + 1000
          print(x)
          print(y)
        \end{lstlisting}
        What do you think the following prints?
        \begin{lstlisting}[language=Python, basicstyle=\footnotesize\ttfamily]
          x = 123
          x = x*2 + 1000
          print(x)
        \end{lstlisting}
        %
        What do you think the following prints?
        \begin{lstlisting}[language=Python, basicstyle=\footnotesize\ttfamily]
          x = 123
          x = x*2 + 1000
          print(x)
        \end{lstlisting}
        %
        Here's a more involved example.  See what it's trying to do?
        \begin{lstlisting}[language=Python, basicstyle=\footnotesize\ttfamily]
          target = 20.0
          x = 4.0
          x = x - 0.01*(x*x*(x - target/x)) 
          x = x - 0.01*(x*x*(x - target/x)) 
          x = x - 0.01*(x*x*(x - target/x)) 
          x = x - 0.01*(x*x*(x - target/x)) 
          x = x - 0.01*(x*x*(x - target/x)) 
          x = x - 0.01*(x*x*(x - target/x)) 
          y = x*x
          message = str(x) + ' squares to ' + str(y) + ' which nearly equals ' + str(target) 
          print(message)
        \end{lstlisting}
        If you successfully executed the above code, then you have officially
        done \textbf{iterative optimization}!  Notice the repeated instruction
        \texttt{x=(target/x+x)/2}.  Each time this instruction is executed,
        \texttt{x}'s value gets closer to the square root of \texttt{target}'s
        value.  6.86x features this theme of ``repeatedly improve in order to
        optimize'' in the guises of \textbf{gradient descent} and of
        \textbf{expectation maximization}.  In fact, the code above is gradient
        descent on the loss function $\ell(x) = (x^2-\text{target})^2$.


      \samsubsubsection{ints, floats, strings, bools}
        %We speak of a variable having a value.
        %A fixed variable may have different values at different times.
        The variable values we work with might be numbers,
        or pieces of text, or answers to a true-false question, or
        something else.  The kind of thing a value is is called that value's
        \textbf{type}.
        %
        We use the jargon \textbf{string} or \textbf{str}
        for the type of thing that chunks of text are.  For example, much as
        \texttt{3.141} is a number, \texttt{'hello, world!'} is a string.
        %
        The jargon \textbf{bool} denotes the type of thing that can answer
        a true-or-false question.  For example, while the value \texttt{'True'}
        is a string, the value \texttt{True} is a bool.  So is \texttt{False}.
        Those are the only two possible bools.
        %
        We'll distinguish two kinds of numeric value: values who by dint of
        their type are integers, and values that might or might not be
        integers.  The jargon word for the former's type is \textbf{int};
        for the latter's, \textbf{float}.\bcirc\marginnote{%
           \blarr Historically, \textbf{int} abbreviates \emph{integer} and
           \textbf{float} stands for \emph{floating decimal point} (except with
           binary instead of decimal).  The poetic image is of a decimal point
           that can drift around as if in a sea:
           \texttt{31.41}, \texttt{3.141}, \texttt{0.3141},
           \texttt{0.03141}, etc.
           %
           Floats thus stand in contrast to ``banker's'' numbers, which do not
           track differences smaller than one-hundredth of a dollar.
           %
           %Floats can express very large and very small quantities such as
           %\texttt{314100000000} and \texttt{0.000000000003141}.
        }  For example, when we write \texttt{5} or \texttt{686}, the
        interpreter understands those values to be ints; when we write
        \texttt{5.0} or \texttt{0.003141}, the interpreter understands those
        values to be floats.

        Different types support different operations and activities.  For
        example, it makes sense to multiply two floats but not two strings.
        And if \texttt{x} is a variable containing a value of type
        \texttt{int}, then it makes sense to access a string's \texttt{x}th
        character.  But that wouldn't make sense with \texttt{int} replaced by
        \texttt{float}.

        Here's a flash summary of several common operations: 

        \begin{table}[h]
          \begin{tabular}{cccc}
            \textsc{operation}  &\textsc{input type(s)}           &\textsc{output type} &\textsc{how to write it}               \\
            \hline
            add                 &\texttt{int}   , \texttt{int}    &\texttt{int}         &\texttt{x+y}                           \\
            add                 &\texttt{int}   , \texttt{float}  &\texttt{float}       &\texttt{x+y}                           \\
            add                 &\texttt{float} , \texttt{int}    &\texttt{float}       &\texttt{x+y}                           \\
            add                 &\texttt{float} , \texttt{float}  &\texttt{float}       &\texttt{x+y}                           \\
            literal int 5       &                                 &\texttt{int}         &\texttt{5}                             \\
            literal float 5.0   &                                 &\texttt{float}       &\texttt{5.0}                           \\
            \hline
            count characters    &\texttt{str}                     &\texttt{int}         &\texttt{len(s)}                        \\
            $n$th character     &\texttt{str}   , \texttt{int}    &\texttt{str}         &\texttt{s[n]}                          \\
            concatenate         &\texttt{str}   , \texttt{str}    &\texttt{str}         &\texttt{left\_str+right\_str}          \\
            literal string '5'  &                                 &\texttt{str}         &\texttt{'5'}                           \\
            \hline
            round toward zero   &\texttt{float}                   &\texttt{int}         &\texttt{int(f)}                        \\
            promote to float    &\texttt{int}                     &\texttt{float}       &\texttt{float(n)}                      \\
            render              &\texttt{int}                     &\texttt{str}         &\texttt{str(n)}                        \\
            render              &\texttt{float}                   &\texttt{str}         &\texttt{str(f)}                        \\
            parse               &\texttt{str}                     &\texttt{int}         &\texttt{int(s)}                        \\
            parse               &\texttt{str}                     &\texttt{float}       &\texttt{float(s)}                      \\
            \hline
            negation            &\texttt{bool}                    &\texttt{bool}        &\texttt{not x}                         \\
            conjunction         &\texttt{bool}  , \texttt{bool}   &\texttt{bool}        &\texttt{x and y}                       \\
            disjunction         &\texttt{bool}  , \texttt{bool}   &\texttt{bool}        &\texttt{x or y}                        \\
            literal bool False  &                                 &\texttt{bool}        &\texttt{False}                         \\
            \hline
            check equality      &\texttt{str}   , \texttt{str}    &\texttt{bool}        &\texttt{x==y}                          \\
            check equality      &\texttt{int}   , \texttt{int}    &\texttt{bool}        &\texttt{x==y}                          \\
            check at-most       &\texttt{int}   , \texttt{int}    &\texttt{bool}        &\texttt{x<=y}                          \\
            check at-least      &\texttt{int}   , \texttt{int}    &\texttt{bool}        &\texttt{x>=y}                          \\
            check at-most       &\texttt{float} , \texttt{float}  &\texttt{bool}        &\texttt{x<=y}                          \\
            check at-least      &\texttt{float} , \texttt{float}  &\texttt{bool}        &\texttt{x>=y}                          \\
          \end{tabular}
          \caption{%
            Python supports all of addition, subtraction,
            multiplication, and exponentiation: we use the symbols
            \texttt{+}, \texttt{-}, \texttt{*}, \texttt{/}, and \texttt{**}.
            For brevity we listed only addition in this table.
          }
        \end{table}

        That's a lot to take in.  Let's digest by seeing these operations
        in action:
        \begin{lstlisting}[language=Python, basicstyle=\footnotesize\ttfamily]
          my_string = 'dear world,' 
          my_string = my_string + ' my favorite number is ' + str(389)
          print(my_string)
        \end{lstlisting}
        And
        \begin{lstlisting}[language=Python, basicstyle=\footnotesize\ttfamily]
          s = 'hi, world!' 
          my_message = 'my string s has ' + str(len(my_string)) + 'many characters,'
          my_message = my_message + ' of which the first and last couple are:'
          my_message = my_message + s[0] + s[1] + '...' + s[8] + s[9]
          print(my_message)
        \end{lstlisting}
        And
        \begin{lstlisting}[language=Python, basicstyle=\footnotesize\ttfamily]
          x = 3 * 5.01
          y = 3.01 * 5
          s = 'abcdefghijklmnopqrstuvwxyz'
          is_nearly_equal = (y-0.1 < x) and (x < y+0.1) 
          print(is_nearly_equal)
        \end{lstlisting}

        \newpage
      \samsubsubsection{input}
      \samsubsubsection{conditionals}% also mention trivalent CHOICE operator?
      \samsubsubsection{iteration}

        %
        \marginnote{%
          % for-range idiom; break, continue, else in loops 
        }
      \samsubsubsection{nesting, scope, indentation}
        \attnsam{skip, abort}

      \samsubsubsection{remarks on `types'}
      \samsubsubsection{data in aggregate}%container structures}
        \attnsam{lists, strings, numpy arrays} % no tuples
        \attnsam{dictionaries} % no sets; mention cacheing (and hence also global scope??)
        \attnsam{comprehensions}
        \attnsam{functional idioms}

      \samsubsubsection{example: the restaurant bill}
        Let's write a program that takes a list of prices as command line input
        and displays a grand total including with $6\%$ tax and a variety of
        tips: $9\%,12\%,15\%,18\%,21\%$.  We'll write the program in two
        styles.  Here is a more conceptual style: 
        \begin{lstlisting}[language=Python, basicstyle=\footnotesize\ttfamily]
          import sys
          prices = [float(arg) for arg in sys.argv[1:]] 
          total = sum(prices) 

          with_tax = total * (1.0 + 0.06)
          print('raw price:', total   , 'dollars')
          print('with tax:' , with_tax, 'dollars')
          for tip in range(9, 24, 3):
            with_tax_and_tip = total * (1.0 + 0.06 + tip/100.0) 
            print('tip =', tip, 'percent', '-->', with_tax_and_tip, 'dollars')
        \end{lstlisting}
        We can say the same thing more explicitly as follows: 
        \begin{lstlisting}[language=Python, basicstyle=\footnotesize\ttfamily]
          import sys
          prices = []
          for arg in sys.argv[1:]:
            prices.append(float(arg))
          total = 0.0
          for price in prices:
            total = total + prices

          print('raw price:', total               , 'dollars')
          print('with tax:' , total * (1.0 + 0.06), 'dollars')
          for tip in range(9, 24, 3):
            print('tip =', tip, 'percent', '-->', total * (1.0 + 0.06 + tip/100.0), 'dollars')
        \end{lstlisting}

      \newpage
      \samsubsubsection{example: equation solver}
        Let's write a program that takes a one-variable math equation as a
        command line argument and displays a solution.  We'll restrict
        ourselves to polynomial equations such as \texttt{xxxxx=xx+xx+1+1} (our
        way of writing $x^5=2x^2+2$).  That is, we'll assume the equation has
        no spaces and consists of only the characters \texttt{x}, \texttt{=},
        \texttt{+}, \texttt{1}, \texttt{0}.  Two more valid equations are
        \texttt{xxx+xx+x+1=0} and \texttt{x=xx}.

        \begin{lstlisting}[language=Python, basicstyle=\footnotesize\ttfamily]
          # step 0: read the equation-to-solve from the command line
          import sys
          import numpy as np
          equation = sys.argv[1]

          # step 1: check whether the equation holds for each x in a large
          #         range with fine spacing; print and exit if for one of these
          #         x's the equation holds to tolerance plus-minus 0.1 
          for x in np.arange(-100.00, +100.01, 0.01): 

            # step 1a: initialize accumulator variables that will eventually
            #          represent the values of the equations's left hand and
            #          right hand sides
            index = 0
            left_hand_side = 0.0
            right_hand_side = 0.0
            term = 1.0

            # step 1b: compute the left hand side's numeric value; 
            #          the left hand side ends the moment we see an equals sign
            while equation[index] != '=':
              if   equation[index] == '0': term = term * 0.0
              elif equation[index] == '1': term = term * 1.0
              elif equation[index] == 'x': term = term * x 
              else:
                left_hand_side = left_hand_side + term
                term = 1.0

            # step 1c: skip the equals sign
            index = index + 1

            # step 1d: compute the right hand side's numeric value;
            #          the right hand side ends when our equation ends 
            while index != len(equation):
              if   equation[index] == '0': term = term * 0.0
              elif equation[index] == '1': term = term * 1.0
              elif equation[index] == 'x': term = term * x 
              else:
                right_hand_side = right_hand_side + term
                term = 1.0

            # step 1e: check whether the left and right hand sides are nearly
            #          equal.  if they are, report and exit 
            difference = left_hand_side-right_hand_sided
            if -0.1<difference and difference<+0.1:
              print('x =', x, 'solves', equation, 'with error only', difference)
              sys.exit()

          # step 2: report that the search for a solution was unsuccessful.
          #         since step 1e exits when a solution is found, the
          #         interpreter will get to step 2 only if no solution is found
          print('no solution found to desired tolerance')
        \end{lstlisting}

    \newpage
    \samsubsection{structuring code}
      \samquote{
        If I have not seen as far as others, it is because giants were standing
        on my shoulders.
      }{hal abelson}

      \samsubsubsection{routines}
        %\attnsam{mention SCOPE somewhere (perhaps earlier than this passage?)}
        %\attnsam{lambdas}
        \attnsam{\texttt{def}d functions (ordinary args, return)}
        \attnsam{interfaces: (kwargs; None as default return value; sentinels; more)}
        \attnsam{code architecture and hygiene}
      \samsubsubsection{more input/output}
        \attnsam{print formatting and flushing}
        \attnsam{basic string manipulations (strip, join, etc)}
        \attnsam{file io}
        \attnsam{command line arguments}
        \attnsam{example: read csv}
        %
        \attnsam{random generation and seeds??}
      \samsubsubsection{how not to invent the wheel} %modules: os, numpy, matplotlib}
        \attnsam{matplotlib.plot}
        \attnsam{numpy}
        \attnsam{os}
        \attnsam{package managers etc}

    \newpage
    \samsubsection{numpy and friends}
      \samquote{
          ... it was unthinkable [to] question the validity of symmetries under
          ``space inversion'' .... It would have been almost sacrilegious to do
          experiments to test such unholy thoughts.
      }{wu chien shiung}

      \samsubsubsection{arrays aka tensors}
        \attnsam{}
        \attnsam{}
        \attnsam{}
        \attnsam{}
      \samsubsubsection{images: viewing, representing, rendering}
      \samsubsubsection{gradient descent done manually}
      \samsubsubsection{gradient descent in pytorch}
      \samsubsubsection{gradient descent in tensorflow}

\end{document}

